\section{Introduction}
Websites implement cookie banners to allow users to either consent to or reject third-party cookie tracking and manage their privacy settings. After tighter legislation came into force, namely EU's General Data Protection Regulation (GDPR) and UK's Data Protection Act 2018 (DPA), more and more websites have adopted such notices, making cookie banners a part of users' everyday life. 

In theory, cookie banners (a.k.a. cookie notices) exist to empower users by informing them about tracking activity and allowing them to opt out if they wish to. However, real-world implementations of cookie banners appear to be a nuisance more than anything else~\cite{kulyk2018website}. Many websites design their notices to make opting out extremely hard, or remove the option completely as previous studies found~\cite{nouwens2020dark}. Furthermore, pre-selected options that nudge users towards privacy-intrusive choices are rife and significantly impact user behaviour~\cite{utz2019informed}. 
Both EU and UK regulators have clearly identified such  practices non-compliant with the GDPR and the DPA, including consent not being explicit and cookie rejection not being as easy as acceptance (See e.g.\ the EU's 2002 ePrivacy Directive~\cite{eprivacy-directive-2002}, the 2020 European Data Protection Board guidelines~\cite{edpb-guidelines-2020}, and discussions by Nouwens et al.~\cite{nouwens2020dark}). 
Yet, even if users manage to navigate around the maze of options and select their privacy settings, their choices are more likely to be ignored entirely as a study of Consent Management Providers (CMPs) deployed in European websites observed~\cite{matte2020cookie}. Worryingly, we have seen such \say{dark patterns} employed by big-tech, such as Facebook~\cite{council2018deceived}. 

The insight we have about the cookie banner landscapes and how they have changed as a result of legislation is mainly based on the analyses carried out on samples of high-traffic websites. Although such studies provide valuable information on how popular websites implement cookie banners, a natural question to ask is how well such observations generalise if a more comprehensive sample including lower-traffic websites is analysed. 
In this work, we aim to take a step towards investigating this question in the UK and Greece web landscapes. 

We set out to establish the types of cookie banners with which users have to interact  on a daily basis. 
Moreover, we will explore the distribution and availability of choices provided to users through cookie banner implementations. 
Using purpose-built software and with the aid of OpenWPM~\cite{englehardt2016census}, we collected, categorised and analysed more than 7,500 cookie banners from more than 17,000 websites across Greece and the UK. We discuss our findings which interestingly in some cases substantially differ with previous results in the literature. Our results therefore is a step towards developing a more clear and comprehensive understanding of the cookie banner landscape in the two countries. 

We consider Greece and the UK because of our familiarity with the respective languages and our hope that the comparison between the two provides interesting insight. On the one hand, websites in both countries adhere to very similar data protection laws. On the other hand however, the two countries vastly differ in their population size and their citizens' use of internet services~\cite{desi-report-2020}. 
