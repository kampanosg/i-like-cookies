
\section{Data and Results}

In this section we specify the dataset and discuss our findings. 

\subsection{The Collected Dataset}
\label{sec:dataset}

\paragraph{Websites.}
The Tranco list contains a total of 1\,M websites. From there, 3,446 are \texttt{.gr} websites and 18,768 are \texttt{.uk} ones. The additional country-specific lists provided an additional 674 websites: 40 additional Greek websites from TopGR (\url{https://topgr.gr}) and 634 additional UK websites from Kadaza (\url{www.kadaza.co.uk}) and Finder (\url{www.finder.com/uk}). Furthermore, we removed 125 Greek and 305 UK websites from the dataset as they were not accessible. In total, the initial dataset contained 3,361 Greek and 19,097 UK available websites. 

We checked each website to determine whether they allow crawlers. The Robots Exclusion Standard parser yielded 3,157 Greek (93\%), and 15,410 UK (69\%) websites that allowed crawling. Then the Terms of Service parser determined that 3,087 Greek (91\%) and 14,650 UK (65\%) websites permitted our study to crawl them. Table~\ref{tab:data_websites}, summarises the breakdown of our dataset.

\begin{table}[t]
    \centering
    \caption{Breakdown of the number of websites per country that are included (by source) and excluded (by reason)}
    \begin{tabular}{@{}l@{\quad}r@{\quad}r@{\quad}r@{}}
    \toprule
        & \textbf{Greece}   & \textbf{UK}       & \textbf{Combined} \\ 
    \midrule
    Included from Tranco 
        & 3,446             & 18,768            & 22,214            \\
    Included from country-specific lists 
        & 40                & 634               & 674               \\
    Excluded due to unavailability 
        & $-$125            & $-$305            & $-$430              \\

    Excluded due to \texttt{robots.txt} 
        & $-$204            & $-$3,687          & $-$3,891          \\
    Excluded due to Terms of service 
        & $-$70             & $-$760            & $-$830            \\
    \midrule 
    \textbf{Total studied} 
        & \textbf{3,087}    & \textbf{14,650}   & \textbf{17,737}   \\ 
    \bottomrule
    \end{tabular}
    \label{tab:data_websites}
\end{table}

\paragraph{OpenWPM.}

Successfully crawling websites for their cookie banners was possible by extending OpenWPM. In addition to cookie banners, OpenWPM also collected  more than 15\,M data points in Greece and the UK. This included information about the HTTP Requests and Responses (3.9\,M), scripts that a website loads (7.6\,M), and cookies stored in a user's web browser (2.3\,M). 

\paragraph{Viking.}
Collecting cookie banners for thousands of websites is a highly computing-intensive task, requiring over 24~hours for a complete crawl in Greece for example, even with parallel crawlers. To overcome this limitation we utilised University of York's Viking cluster\footnote{See \url{www.york.ac.uk/it-services/research-computing/viking-cluster}}, a high-performance computing cluster with 173 nodes, 42\,TB of memory, and 7024 Intel cores. While only using a fraction of Viking's resources (128\,GB of memory and 32 cores) the crawl was completed in 8~hours for Greece and just over 36~hours for the UK.

\subsection{Findings}
\subsubsection{\ref{rq:prevalence}: Prevalence depends on sample size.}
Our findings show that almost half of the websites we surveyed display a cookie banner. More specifically, around 48\% of Greek and 44\% of the UK websites included a cookie notice. 

\begin{table}[t]
    \centering
    \caption{Comparison of measured cookie banner prevalence rates. Sample sizes are approximates. Ranges indicate two methods of measurement.}
    \begin{tabular}{@{}l@{\quad}c@{\quad}c@{\quad}c@{\quad}c@{\quad}c@{}}
        \toprule
        \textbf{Study} & \textbf{Year} & \multicolumn{2}{c}{\textbf{Sample Size}} & \multicolumn{2}{c}{\textbf{Prevalence}} \\ 
        & \textbf{Conducted} & \textbf{UK} & \textbf{GR} & \textbf{UK} & \textbf{GR} \\ 
        \midrule 
        Degeling et al.~\cite{degeling2018we} & 2018 & 500 & 500 & 67--82\% & 60--69\% \\ 
        van Eijk et al.~\cite{eijk2019impact} & 2019 & 100 & 100 & 52\% & 29\% \\ 
        This work & 2020 & 14,000 & 3,000 & 44\% & 48\% \\ 
        \bottomrule
    \end{tabular}
    \label{tab:prev-comp}
\end{table}


When comparing our results with previous works of Degeling et al.~\cite{degeling2018we} and van Eijk et al.~\cite{eijk2019impact}, an interesting pattern emerges. As shown in Table~\ref{tab:prev-comp}, although both van Eijk et al.'s and our data collection were conducted after that of Degeling et al., we report lower prevalence than that of the earlier study. This is at odds with the reasonable expectation that the prevalence of cookie banners does not decrease substantially over time. What can explain this discrepancy is the sample size factor. Our results demonstrate that the observed prevalence depends on the size of the sample. That is, although the observed rates might rise initially as samples are expanded from the top hundred to a few hundred websites in each country, further expansion to a few thousands results in a decrease in observed prevalence rates. Hence, our results show that studies with smaller sample sizes might not provide an accurate representation of the big picture. 

Using the additional data collected by OpenWPM, we found that 61\% of Greek and 70\% of UK websites store at least one third-party cookie on their user’s browser. This suggests that around 13\% of Greek and 26\% UK websites have yet to comply with the GDPR or DPA, respectively, as shown in Fig.~\ref{fig:prevalence_cookie_banners_tps}. 


\begin{figure}
    \centering
    \renewcommand{\bcfontstyle}{}
    \begin{bchart}[step=10,max=80, unit=\%]
        \bcbar[text=Stores 3rd-Party Cookies, color=orange]{61}
        \bclabel{Greece}
        \bcbar[text=Displays cookie banner, color=green]{48}
        \smallskip
        \bcbar[text=Stores 3rd-Party Cookies, color=orange]{70}
        \bclabel{UK}
        \bcbar[text=Displays cookie banner, color=green]{44}
    \end{bchart}
    \caption{Websites that store 3rd-party cookies and display a cookie banner.}
    \label{fig:prevalence_cookie_banners_tps}
\end{figure}



\subsubsection{\ref{rq:avg_options}: Direct opt-outs are rare.}
The distribution of the number of options cookie banners in our dataset provide is depicted in Fig.~\ref{fig:avg_options}. 
As the figure shows, the most prevalent number of options in both countries is two. 
The mean number of options is 2.1 for Greece and 1.8 for the UK. 
The median number of options is 2 for both countries. 
This is in agreement with van Eijk et al.'s finding that the median number of choices in the top 100 popular websites that have a cookie banner in both countries was two~\cite{eijk2019impact}. 

\pgfplotstableread[row sep=\\,col sep=&]{
    interval    & Greece & UK \\
    No Options  & 0.3    & 1 \\
    1 Option    & 22     & 29  \\
    2 Options   & 57     & 58 \\
    3 Options   & 20     & 12 \\
    4 Options   & 4      & 0.7 \\
}\categories

\begin{figure}[t]
    \centering
    \begin{tikzpicture}
        \begin{axis}[
                ybar,
                bar width=.6cm,
                width=\textwidth,
                height=.5\textwidth,
                symbolic x coords={No Options,1 Option,2 Options,3 Options, 4 Options},
                xtick=data,
                nodes near coords,
                nodes near coords align={vertical},
                ymin=0,ymax=100,
                ylabel={Cookie Banner \%},
            ]
            \addplot[fill=blue] table[x=interval,y=Greece]{\categories};
            \addplot[fill=cyan] table[x=interval,y=UK]{\categories};
            \legend{Greece, UK}
        \end{axis}
    \end{tikzpicture}
    \caption{Distributions of cookie banners per number of options in Greece and the UK.}
    \label{fig:avg_options}
\end{figure}

Worryingly, we can see in Fig.~\ref{fig:avg_options} that a considerable proportion of cookie banners provide either no option or only one option to the user. 
This prompts us to look into the distribution of the four categories of privacy options in the cookie banners. 
The results are depicted in Fig.~\ref{fig:priv_categories_breakdown}. 
As the figure shows, although Affirmative options are quite ubiquitous in cookie banners in both countries (Greece: 95\%, UK: 88\%), Negative options are quite rare (Greece: 20\%, UK: 6\%). 
In the upcoming sections we will look further into the exact combinations of options provided by the websites to be able to draw further conclusions. 

\pgfplotstableread[row sep=\\,col sep=&]{
    interval        & Greece & UK    \\
    Affirmative     & 95   & 88  \\
    Negative        & 20   & 6  \\
    Managerial      & 50  & 69 \\
    Informational   & 40  & 20 \\
}\categories

\subsubsection{\ref{rq:no_options}: Most cookie banners nudge towards privacy-intrusive choices.}
Considering all the ways the four categories of options that we have discussed may appear in a cookie banner results in 16 possible combinations. 
We depict the distributions of all these 16 option combinations in both countries in Table~\ref{tab:combos-all}. The combinations are coded with abbreviations in the table, e.g. \texttt{A-M-} stands for the combinations in which at least an Affirmative choice is present, Negative absent, Managerial present, and Informational absent. 

As Table~\ref{tab:combos-all} shows, by far the most prevalent combination is that of Affirmative and Managerial options (i.e. \texttt{A-M-}), with other combinations including an Affirmative option but excluding a Negative option (i.e. \texttt{A-MI}, \texttt{A--I}, and \texttt{A---}) following in terms of prevalence in both countries. 
This shows that at least 75\% of cookie banners in Greece and 82\% in the UK explicitly nudge their users towards accepting cookies. 

\begin{figure}[t]
    \centering
    \begin{tikzpicture}
        \begin{axis}[
                ybar,
                bar width=.6cm,
                width=\textwidth,
                height=.5\textwidth,
                symbolic x coords={Affirmative,Negative,Managerial,Informational},
                xtick=data,
                nodes near coords,
                nodes near coords align={vertical},
                ymin=0,ymax=100,
                ylabel={Cookie Banner \%},
            ]
            \addplot[fill=blue] table[x=interval,y=Greece]{\categories};
            \addplot[fill=cyan] table[x=interval,y=UK]{\categories};
            \legend{Greece, UK}
        \end{axis}
    \end{tikzpicture}
    \caption{The proportion of cookie banners in Greece and the UK providing each type of option.}
    \label{fig:priv_categories_breakdown}
\end{figure}

Going beyond nudging, as Nouwens et al. argue~\cite{nouwens2020dark}, implicit consent and reject not being as easy as accept are both violations of GDPR and DPA. Let us now consider the 16 combinations against these two criteria. 

For explicit consent, one requires at least an Affirmative or a Managerial option to be present so that the user can register their consent explicitly through one of these options. Hence, all combinations without any of these two options (i.e. \texttt{-N-I}, \texttt{-N--}, \texttt{---I}, and \texttt{----}) represent cookie banners that are violating this criterion. Hence, our results show that at least 16 Greek and 129 UK websites are non-compliant with GDPR and DPA since they do not provide the means for their users to register their explicit consent to the use of cookies. These constitute around 1\% of Greek and 2\% of UK websites with cookie banners. 

The proportions of websites not providing an explicit consent option discussed above are large under-estimations since consent is not necessarily explicit in other combinations. More specifically, in our Affirmative category, apart from terms such as \say{accept} and \say{I agree} that clearly indicate consent, there are many other terms with less clear meaning such as \say{close}, \say{continue}, and \say{dismiss}. These less clear terms roughly constitute around one sixth of all of the observed Affirmative options. We do not believe that such terms are sufficient to indicate explicit consent and hence estimate the level of non-compliance in terms of explicit consent to be around 15\%. 

The situation is much worse if the relative ease of Affirmative and Negative options are considered. Any combination with an Affirmative choice but without a Negative one (i.e. \texttt{A-MI}, \texttt{A-M-}, \texttt{A--I}, and \texttt{A---}) clearly does not provide a negative option as easily accessible as an Affirmative one. Furthermore, any cookie banner that only includes an Informative option or no option (i.e. \texttt{---I} and \texttt{----}) is defaulting on acceptance of cookies if the user navigates away from the cookie banner to interact with the website, hence not providing any means for the user to register their lack of consent. Therefore, all of these combinations do not satisfy the criterion either. This means that overall, our results demonstrate that at least 76\% of Greek and 84\% of UK cookie banners violate the GDPR and DPA in that they do not provide their users with a Negative option as easily accessible as an Affirmative one. 

\begin{table}[t]
    \centering
    \caption{Distribution of the 16 combinations of Privacy Options in Greece and the UK, highlighting those that directly violate the GDPR and the Data Protection Act 2018. \texttt{A}: Affirmative, \texttt{N}: Negative, \texttt{M}: Managerial, \texttt{I}: Informational.}
    \begin{tabular}{@{}c@{\qquad}r@{\quad}r@{\qquad}c@{\quad}c@{}}
    \toprule
        \textbf{Combination} & 
        \textbf{GR} & 
        \textbf{UK} & 
        \textbf{\begin{tabular}[c]{@{}c@{}}Consent \\ Explicit\end{tabular}} & 
        \textbf{\begin{tabular}[c]{@{}c@{}}Accept as easy \\ as Reject\end{tabular}} \\ 
        \midrule
        \texttt{ANMI}   & 4\%       & 1\%       &       &       \\
        \texttt{ANM-}   & 5\%       & 3\%       &       &       \\
        \texttt{AN-I}   & 10\%      & $<$1\%    &       &       \\
        \texttt{AN--}   & 2\%       & 1\%       &       &       \\
        \texttt{A-MI}   & 5\%       & 8\%       &       & No    \\
        \texttt{A-M-}   & 32\%      & 47\%      &       & No    \\
        \texttt{A--I}   & 21\%      & 8\%       &       & No    \\
        \texttt{A---}   & 17\%      & 19\%      &       & No    \\
        \texttt{-NMI}   & $<$1\%    & 0\%       &       &       \\
        \texttt{-NM-}   & $<$1\%    & $<$1\%    &       &       \\
        \texttt{-N-I}   & 0\%       & 0\%       & No    &       \\
        \texttt{-N--}   & 0\%       & 0\%       & No    &       \\
        \texttt{--MI}   & $<$1\%    & 1\%       &       &       \\
        \texttt{--M-}   & 4\%       & 9\%       &       &       \\
        \texttt{---I}   & 1\%       & 1\%       & No    & No    \\
        \texttt{----}   & $<$1\%    & 1\%       & No    & No    \\ 
    \bottomrule
    \end{tabular}
    \label{tab:combos-all}
\end{table}


Degeling et al.~\cite{degeling2018we} report their observations of a several types of cookie banners of top 500 websites in Greece and the UK. Three of their categories can be roughly comparable to collections of combinations we report. 
Cookie banners with \say{no options} in their work roughly correspond to combinations with neither an Affirmative nor a Negative option (i.e. \texttt{--??} where \texttt{?} is a wildcard). They report around 20\% and 40\% for this category (estimated from~\cite[Figure~5(a)]{degeling2018we}) compared to our 5\% and 12\% respectively for Greece and the UK. 
Cookie banners with \say{confirmation only} in their work roughly correspond to combinations with an Affirmative but not a Negative option (i.e. \texttt{A-??}). They report around 65\% and 35\% for this category (estimated) compared to our 75\% and 82\%. 
Cookie banners with a \say{binary} choice in their work roughly correspond to combinations with both an Affirmative and a Negative option (i.e. \texttt{AN??}). They report around 4\% and 5\% for this category (estimated) compared to our 20\% and 5\%. 
These comparisons show that observed practices may substantially vary between observations of smaller and larger sample sizes. 

Limiting their attention to cookie banners provided by the 5 most popular CMPs in the UK, Nouwens et al. found around 75\% violating the \say{reject as easy as accept} criterion~\cite{nouwens2020dark}. 
Our analysis gives the rate of at least 84\% for the violation of this criterion showing that the situation is much worse when a larger set of websites are considered. 

In addition to privacy options, cookie banners usually contain a concise textual description as well. The text's primary function is to inform users why cookies are used and how they may affect them. This text is usually considerably shorter compared to the full Privacy Policy of the website. 
Examples of such texts can be seen in Fig.~\ref{fig:cookie_banners}.

The average length of cookie banner texts in Greek websites was 66 words, slightly longer than the UK average of 52 words. 

Employing the term frequency--inverse document frequency (TF-IDF) formula to identify the most prominent terms in the cookie banner text corpus, we found that the most prominent terms in Greece and the UK are quite similar and dominated by terms with an apparent positive connotation such as \say{best}/\say{better}/\say{$\kappa \alpha \lambda \acute{\upsilon} \tau \varepsilon \rho \eta$}, \say{ensure}, and \say{experience}/\say{$\varepsilon \mu \pi \varepsilon \iota \rho \acute{\iota} \alpha$}. 

In fact, none of the top 50 prominent terms in either country (available from the repository) appear to have a negative connotation, whereas terms with a positive connotation such as \say{improve}/\say{$\beta \varepsilon \lambda \tau \iota \acute{\omega} \sigma \varepsilon \iota$} and \say{enhance} constitute a considerable proportion of the list of terms. 

To get a more comprehensive view of the connotations relayed by cookie banner texts in the UK, we performed an automated sentiment analysis of the words used in all UK banner texts using NRCLex~\cite{nrclex}. 
The analysis found that a generally positive emotional affect was present in around 80\% of the banner texts, whereas a generally negative affect was present in only around 14\%. 
Besides, an overwhelming majority (of more than 9 in 10) of the texts with a negative affect also had a positive effect present as well. 
Looking at more specific emotional affects, trust and joy are among the most prevalent, present in around 66\% and 46\% of the texts, respectively. 
The prevalence of general and specific emotional affects is shown in Fig.~\ref{fig:sentiment-analysis}. 

\pgfplotstableread[row sep=\\,col sep=&]{
    Affect          & UK \\
    Positive        & 80 \\
    Negative        & 14 \\
}\ukpol

\pgfplotstableread[row sep=\\,col sep=&]{
    Affect          & UK \\
    Fear            & 19 \\
    Anger           & 4 \\
    Anticipation    & 52 \\
    Trust           & 66 \\
    Surprise        & 4 \\
    Sadness         & 8 \\ 
    Disgust         & 1 \\
    Joy             & 46 \\
}\ukaff

\begin{figure}[t]
    \centering
    \begin{minipage}[t]{0.3\textwidth}
        \begin{tikzpicture}[baseline=(current bounding box.north)]
            \begin{axis}[
                    ybar,
                    bar width=.55cm,
                    width=\textwidth,
                    height=20em,
                    symbolic x coords={Positive, Negative},
                    xtick=data,
                    nodes near coords,
                    nodes near coords align={vertical},
                    ymin=0,ymax=100,
                    ylabel={Affect presence \%},
                    x tick label style={rotate=45,anchor=east},
                    enlarge x limits=0.5
                ]
                \addplot[fill=cyan] table[x=Affect,y=UK]{\ukpol};
                \legend{UK}
            \end{axis}
        \end{tikzpicture} 
    \end{minipage}
    \hfill
    \begin{minipage}[t]{0.6\textwidth}
        \begin{tikzpicture}[baseline=(current bounding box.north)]
            \begin{axis}[
                    ybar,
                    bar width=.55cm,
                    width=\textwidth,
                    height=20em,
                    symbolic x coords={Trust, Anticipation, Joy, Fear, Sadness, Surprise, Anger, Disgust},
                    xtick=data,
                    nodes near coords,
                    nodes near coords align={vertical},
                    ymin=0,ymax=100,
                    ylabel={Affect presence \%},
                    x tick label style={rotate=45,anchor=east},
                    % enlarge x limits=0.2
                ]
                \addplot[fill=cyan] table[x=Affect,y=UK]{\ukaff};
                \legend{UK}
            \end{axis}
        \end{tikzpicture}
    \end{minipage}
    \caption{Distributions of the emotional affects in the UK cookie banner text corpus}
    \label{fig:sentiment-analysis}
\end{figure}

The automated term prominence and sentiment analyses above suggest that websites tend to give a one-sided description of cookie usage, namely that it enhances browsing experience, conveniently leaving out that cookies can be used for tracking. This is in line with previous manual analyses of smaller samples, e.g.\ that of Utz et al.~\cite{utz2019informed}, that found similar biases in cookie banner texts. 


\subsubsection{\ref{rq:manage_options_count}: Managing trackers is more prevalent than opting out.}
We aimed to determine whether websites allow their visitors to manage their privacy settings from the cookie notice. Our results show that Managerial options in cookie banners are significantly more prevalent compared to the Negative ones (See Fig.~\ref{fig:priv_categories_breakdown}). More specifically, 59\% of Greek and 69\% of UK cookie banners offer a Managerial option compared to 20\% and 6\%, respectively, with a Negative one. Hence, users in both countries are several times more likely to be given an option to manage their cookies than an option to decline them. 

\subsubsection{\ref{rq:distribution}: Users in both countries lack real choice, but practices vary.}
The results discussed in the previous sections show that Greek and UK users face a largely similar landscape in terms of the prevalence of cookie banners, widespread deployment of third-party cookies, and rampant use of nudging and lack of consent to cookies being much harder to register than consent. 
However, there are some notable differences between the two countries. 
With respect to using third-party cookies but not showing a cookie banner at all, the proportion of websites that are non-compliant with regulations in the UK is almost double that in Greece. Besides, cookie banner in Greece tend to provide slightly higher number of options. 

Looking at the option types, affirmative options are prevalent in both, but negative options are much scarcer in the UK. From the other two types, managerial options are more prevalent in the UK and informational ones more in Greece. Looking at the option combinations, most have similar prevalence with three exceptions: although \texttt{AN-I} can be found in around 10\% of banners in Greece, it is very rare in UK; similarly, \texttt{A--I} has a much higher share in Greece (21\%) than in the UK (8\%); on the other hand, \texttt{A-M-} is found in almost half of the cookie banners in the UK, but only in about a third in Greece. 