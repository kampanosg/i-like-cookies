\documentclass[../main.tex]{subfiles}

\begin{document}
The increase in internet usage and online services such as internet banking and e-commerce has brought with it extended online-tracking and data gathering. The European Union and the UK have reacted to this new threat by passing laws such as the GDPR and the Data Privacy Act 2018. Such legislation forces websites to disclose their tracking activities and the parties that have access to the collected information. 

In response to the legislation, websites implemented Cookie Banners $-$ pop-ups that appear when a user visits a website for the first time. Since Cookie Banners are commonplace, this project has aimed to broaden the understanding of the Cookie Banner landscape, how they affect users and what type of privacy options they offer.

Specifically, this project looked at cookie banners in Greece and the UK since they are both governed by Data Protection laws but they also differ in language and size. More than 14,500 websites were surveyed across both countries. This was done by using OpenWPM \cite{englehardt2016online}, a popular web-privacy measurement framework, which was extended to detect and store Cookie Banners. Furthermore, for the purposes of this study, a number of novel methods and techniques were developed that were used to identify popular websites and sanitise and normalise the collected data.

This project collected more than 7,500 cookie banners and over 15 million datapoints from OpenWPM. The results show that although 52\% of websites have implemented Cookie Banners, more than 12\% of the sample does not have one and therefore, not complying with the law. Furthermore, it is evident that websites use a number of dark patterns to nudge users towards privacy-intrusive choices. Specifically, only 5\% of websites across both countries have direct opt-out links. Moreover, the majority of websites nudge users into accepting cookies since they \say{improve their browsing experience}.

\end{document}